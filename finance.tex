\documentclass[a4paper,10pt]{article}
\usepackage[utf8]{inputenc}
\usepackage[frenchb]{babel}
\usepackage[T1]{fontenc}

\usepackage{authblk}
\usepackage{frbib}
\usepackage{amsfonts}
\usepackage{amsmath}
\usepackage{amssymb}
\usepackage{array}
\usepackage{url}
\usepackage{hyperref}
\usepackage{textcomp}
\usepackage[all]{hypcap}
\usepackage[labelseparator=endash]{caption}
\usepackage[listofformat=parens]{subfig}
\usepackage{graphicx, color}
\usepackage{listings}
\usepackage{float}
\bibliographystyle{frplain}
\hypersetup{colorlinks,%
            citecolor=black,%
            filecolor=black,%
            linkcolor=black,%
            urlcolor=blue}

\newcommand{\LSTRESET}{
  \lstset{
    language=,
    basicstyle=\color{black},
    identifierstyle=\color{black}
}
}

\setlength{\affilsep}{0.2cm}
\author{Ken Déguernel, François Deslandes}
\affil{Génie Mathématique 5ème année}
\affil{A l'attention de Mme Lehmann}
%\date{4 Décembre 2013\\}
\title{\Huge{Entrée en bourse des réseaux sociaux}\\
\LARGE{Etude comparative Facebook vs. Twitter}\\
\vspace{10mm}
}

\DeclareMathOperator{\Rec}{Rec}
\DeclareMathOperator{\card}{card}

\begin{document}
\maketitle\thispagestyle{empty}
 
\newpage\null\thispagestyle{empty}\setcounter{page}{0}

\newpage
\tableofcontents

\clearpage

\section{Introduction}
% Dire que l'entrée en bourse des réseaux sociaux est un nouveau cas d'étude en finance.
% Parler du fait que ce sont des entreprises qui doivent faire leurs preuves sur le long terme, prouver qu'elles peuvent rapporter de l'argent.
% En bref, que c'est ces entreprises doivent montrer qu'elles ont un business model qui tient la route.

% Les entrées en bourse de Facebook et Twitter sont particulièrement intéressantes puisqu'elles ont conduit à des comportements très différents de la part des marchés financier.
% Les deux entreprises ont malgré tout montré une certaine solidité consécutive à leur entrée en bourse. Cependant, leur business model doit faire ses preuves, particulièrement dans le contexte actuel, et est donc particulièrement sensible aux effets d'annonce.



\section{Présentations}
% Présentation générale des deux réseaux sociaux :
% Utilisateur, Business model, Management.
% Leur actualité :
% Concurrent, nouveautés (facebook on mobile), débat sur la confidentialité des données, etc.
% résultats

\subsection{Facebook}
\subsection{Twitter}

\section{Entrée en bourse}
% Nombre d'action, répartition, prix, controle, facteurs de risque
% Entrée en bourse, réaction, faits marquants
% Evolution consécutive

\subsection{Facebook}
\subsection{Twitter}

\section{Conclusion}
% Twitter n'a pas commis les erreurs de Facebook
% Parler des stratégies (prix, répartition et vente des actions)
% Cas d'étude intéressant pour l'avenir avec la possible entrée en bourse de nouvelle entreprises simialaires ?????


\clearpage
\end{document}